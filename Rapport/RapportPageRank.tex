\documentclass[10pt,a4paper]{article}
\usepackage{inputenc}
\usepackage{graphicx}
\usepackage[french]{babel}
\usepackage[T1]{fontenc}

\usepackage{fontspec}

\usepackage{lmodern}


\usepackage{vmargin}
\usepackage{graphicx}
\usepackage{tabularx}
\setlength{\parindent}{0mm}

\renewcommand{\arraystretch}{2}

%6 pages maximum
\begin{document}

\titlepage{
	\today \vspace{7cm}
	\begin{flushright}\sf\Huge
	{\bfseries LSINF1250} \\[2mm]
	{\bfseries Projet PageRank} \\[1pt]
	{\huge Ranking de réseaux sociaux et page web}
	\end{flushright}
	\ \\[8cm]
	\textbf{Groupe 8} \\
	Denauw Antoine\\
	De Carvalho Borges Antonio
}



\newpage

\section{Procédure Java}
??
\section{Algorithme utilisé}
La méthode que nous avons choisi est l'algorithme PageRank utilisant la PowerMethod comme présenté au cours (cf. Chapitre 10 slide ...) :
\\ Inclure screenshot de l'algo utilisé
\\Explication de l'algorithme

\section{Librairie de calcul matriciel}
Même que vivement conseillé, nous avons pris la décision de ne pas utiliser de librairie spécifique aux calculs matriciel tel que $JAMA$, mais d'implémenter les différentes fonctions par nous même. Ce choix s'est fait dans une optique de ré-adaptation au langage $JAVA$ et de ses règles basiques. %A changer si on décide d'utiliser une librairie JAVA
\\ Pour arriver à reproduire l'algorithme PageRank nous avons du implémenter les fonctions suivantes:
\begin{itemize}
    \item matrix_x_vector : Sers à calculer le vecteur résultant du produit entre une matrice NxN et un vecteur de type Nx1.
    \item degree : Sers à calculer et stocker le degré dans un vecteur de chaque ligne d'une matrice NxN.
    \item multiply (attention à la signature) : Peut servir soit à multiplier un vecteur avec une matrice et un facteur alpha, soit multiplier un vecteur avec un facteur ou bien multiplier deux matrices.
    \item ...
\end{itemize}

\section{Méthode pour déterminer les scores}
??

\section{Annexe}
\subsection{Scores PageRank}

\subsection{Code complet}
%Bien le commenter











\end{document}